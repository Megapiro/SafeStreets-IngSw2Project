\section[Introduction]{\hyperlink{toc}{Introduction}}
	\label{sec:introduction} 
	
	\subsection[Purpose]{\hyperlink{toc}{Purpose}}
		\label{sec:purpose}
		
			This document completely describes the software design and architecture of the SafeStreets system. The entire description we are going to provide is totally based on what we have defined in the \emph{Requirements Analytics and Specification Document} \cite{RASD} but we have tried to maintain an independence between the two documents by giving further descriptions of what we have already said but in a better manner for the aim of this document.\\
		
			It is very important that this document gets completely read and understood before starting with the development of the system and to stick with its details as much as possible as they are thought and engineered as a whole in order to obtain what the system's actual purpose is.
			
	\subsection[Context]{\hyperlink{toc}{Context}}
		\label{sec:context}
		
		SafeStreets is a \emph{crowd-sourced} application that intends to provide a way to notify authorities when a traffic violation occurs. Nowadays systems tend to aim the \emph{crowdsourcing} always more, this way of managing the information in fact reflects different positive aspects that can be used to provide a lot of useful functionalities with a cost that is almost null. Consequentially, several discussions and opinions try to describe which is the best practice to develop a crowd-sourced application. After having considered several alternatives (for example \cite{CB}) we decided to build our system with the aspects we thought to be the most important, grasped from one possibility to the other. As a result we have obtained a solid system based on totally agreeable assumptions that would be really developed in the "real world". 
		
	\subsection[Scope]{\hyperlink{toc}{Scope}}
		\label{sec:scope}
		
		The aim of the system is to achieve with outsourcing a simple way to notify traffic violations and use this information to retrieve interesting data for both its customers: users and authorities. Hence we can think as the main objective of SafeStreets the one of the notification and then describe the other functionalities in the basic and advanced services.\\
		
		The system will describe the notification functionalities in terms of \emph{parking violations} but is thought for further extensions in order to deal with the other types. Users are the clients allowed to report a violation and authorities will be notified whenever a new one will be reported. This process takes place entirely inside the application where both customers, once recognized, will be able to benefit only of the services related to their role. As we have already said SafeStreets is a crowd-sourced application; thanks to this property the big amount of data that is going to be managed will be used to provide additional functionalities based on the processes of mining and crossing this information. We call the \emph{basic functionality} the one related to the data mining used to retrieve statistical data interesting for both the users and authorities. It is important to highlight that this functionality considers the role of each customer in order to provide him a way for retrieving the data with a precise level of visibility. As completely described in the section related to the functionalities provided by the application (\blueRef{sec:generalContext}) the authorities in fact benefit of an additional service that allows them to retrieve the information about the parking violations in the most detailed way (for example with the entire description of an infraction that contains also the data about who reported it and the plate of the vehicle). The \emph{advanced functionality} instead is related to the concept of \textbf{safety} that can be attributed to a street. This functionality is based on the crossing process between the data stored in SafeStreets system related to the violations and the one of the accidents possibly provided by a municipality. The safety of each street our application is able to retrieve is strictly bound with an additional service that wants to find a suggestion for a possible intervention for a street that has been marked as \emph{unsafe}.
			
	\subsection[Glossary]{\hyperlink{toc}{Glossary}}
		\label{sec:glossary}
		
		\subsubsection[Definitions]{\hyperlink{toc}{Definitions}}
			\begin{itemize}
				\item \textbf{DBMS:} is the software system that allows to manage efficiently the data stored in the database of the system.
				
				\item \textbf{Posta Elettronica Certificata:} is a technology that allows to send email with a legal approach.
				
				\item \textbf{Crowdsourcing:} is a sourcing model in which individuals or organizations obtain goods and services from a large, relatively open and often rapidly-evolving group of internet users.
				
				\item \textbf{REST:} is an architectural style for communication based on strict use of HTTP request types.
				
				\item \textbf{UX diagram:} helps to visualize the steps a user takes to complete a task or achieve a goal on a site or app.
			\end{itemize}
		
		\subsubsection[Acronyms]{\hyperlink{toc}{Acronyms}}
			\begin{itemize}
				\item \textbf{RASD:} Requirements Analytics and Specification Document
				\item \textbf{DBMS:} Database Management System
				\item \textbf{EMS:} Email Management System
				\item \textbf{ACI:} Authority Common Interface
				\item \textbf{PEC:} Posta Elettronica Certificata
				\item \textbf{API:} Application Programming Interface
				\item \textbf{IRI:} Image Recognition Interface
				\item \textbf{MI:} Map Interface
				\item \textbf{GPS:} Global Positioning System
				\item \textbf{JSON:} JavaScript Object Notation
				\item \textbf{OS:} Operating System
				\item \textbf{HTTPS:} HyperText Transfer Protocol over Secure Socket Layer
				\item \textbf{TCP:} Transmission Control Protocol
				\item \textbf{IP:} Internet Protocol
				\item \textbf{REST:} Representational State Transfer
				\item \textbf{IOS:} iPhone Operating System	
			\end{itemize}
			
		\subsubsection[Abbreviations]{\hyperlink{toc}{Abbreviations}}
			\begin{itemize}
				\item \textbf{IIT:} Implementation, Integration and Testing
				\item \textbf{R:} Requirement
				\item \textbf{e.g.:} exemply gratia
			\end{itemize}
		
	\subsection[Document Structure]{\hyperlink{toc}{Document Structure}}
		\label{sec:documentStructure}
		
		The document is structured in a double linked way in order to provide an easier and quicker navigation in particular for the several images that can be easily retrieved in this way by going right at the toc and search for a new one.\\
		
		Moreover the document is structured as now briefly described:
		
		\begin{enumerate}
			\item \textbf{\hyperref[sec:introduction]{Introduction}:} gives a first description of the problem and tries to motivate how in this document it is going to be faced first with the purpose and second with the context. The scope is used to point out the main features of the system as they can be always clear in mind while reading the document. The section ends with the glossary.
			
			\item \textbf{\hyperref[sec:architecturalDesign]{Architectural Design:}} this is the core of the entire document. This section starts with a high level description of the architecture of the system and then continues always more in details considering the components and then the interfaces that have been identified. The deployment of the system is also taken into account right after having defined the components and the database view while the interfaces are precised in the next section. The entire section terminates with the precise description of the architectural styles and patterns (sec \blueRef{sec:selectedArchitecturalStylesAndPatterns}) used to define the entire system.
			
			\item \textbf{\hyperref[sec:userInterfaceDesign]{User Interface Design}:} this section aims to a precise description of the interfaces that will conduce the interaction of the clients with the system. As these mockups were already presented in the RASD we want here: first to describe with the UX diagrams how the client navigates the application in order to obtain the functionality he is requiring; second to give a precise identification of each mockup to explain why it needs to be designed in this way in order to provide the functionality it is supposed for.
			
			\item \textbf{\hyperref[sec:requirementsTraceability]{Requirements Traceability}:} once the description of the software design and architecture is clear we identify the relation between the requirements identified in the RASD document \cite{RASD} and the components that allow to realize them.
			
			\item \textbf{\hyperref[sec:iitPlan]{Implementation, Integration and Test Plan}:} finally in this section everything gets together as a plan to develop the entire system needs also to be clearly identified and described. In order to do so, in this section, we first give an overview of the main features we considered to decide the plan. Then, in the next subsection we identify the conditions that are needed to be met before starting with the development process. The utility tree (\blueAutoref{fig:utilityTree}) is presented before the real description of the plan as we thought it would be helpful to understand further motivations provided for the order of the IIT plan. Thanks to the \textbf{Use Relation Hierarchy} diagram of \blueAutoref{fig:useRelationHierarchy} we used the relations to establish the order and then described the plan by matching each phase of \emph{Implementation, Integration and Testing} with the relative number.
			
			\item \textbf{\hyperref[sec:effortSpent]{Effort Spent}:} this section has been used to keep track of the hours spent to complete the document. The first table defines the hours spent together while taking the most important decisions, the seconds instead contains the individual hours.
			
			\item \textbf{\hyperref[sec:references]{References}:} includes all the references used to define the document.
		\end{enumerate}			