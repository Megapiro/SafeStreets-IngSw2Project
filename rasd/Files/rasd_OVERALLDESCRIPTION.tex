\section[Overall Description]{\hyperlink{toc}{Overall Description}}

\subsection[Product Perspective]{\hyperlink{toc}{Product Perspective}}
	Thanks to the general introduction and the scope definition of the system from the previous sections, we are now able to look at our system first from the outside and then from the inside. To deal with this description we are going to see the external interfaces of the system and then the definition of the model's structure in order to interact with them; at the end of the section \textbf{state diagrams} are used to emphasize the dynamic behavior of the most critical classes.
	\subsubsection[System Interfaces]{\hyperlink{toc}{System Interfaces}}
		\label{sec:systemInterfaces}
		SafeStreets offers an interface to its customers to provide them its basic and advanced functionalities. All the data needed to authenticate the users will be managed inside the system as well as the information related to the violations in order to be mined and crossed whenever needed.\\
		
		To accomplish the \hyperref[sec:goals]{\textcolor{blue}{goals}} stated in the \hyperref[sec:introduction]{\textcolor{blue}{introduction}} the application needs also to interact with three main external interfaces as reported in the following picture (\autoref{fig:systemInterfaces}).  
		\vspace{0,3cm}
		
		\begin{figure}[h]
			\centering
			\includegraphics[scale=0.5]{/diagrams/externalInterfaces.png}
			\caption{\label{fig:systemInterfaces}System Interfaces}
		\end{figure}
		
		Two different kinds of interfaces are distinguished in the picture above:
		\begin{itemize}
			\item \textbf{Left hand side:} interfaces that provide functions (by means of APIs) for the system to perform internal operations. 
			
			In particular:
			\begin{itemize}
				\item \textsc{IRI} is used to process the image received from a violation. Whenever a violation is reported the system tries to recognize two distinct things from the picture helped by the additional data provided:
				\begin{itemize}
					\item \textbf{Plate:} the recognition of the plate's number is important to identify the vehicle by checking if it coincides with the one inserted or to add this information when missing. In both the cases when the number can not be identified the violation will be processed without it.
					\item \textbf{Type of vehicle:} is also very important to be recognized as the statistics can also be integrated with this filter. The vehicle will be reported without any type in case the process fails to find one, but we can assume that a shape recognition will be easily carried out by today's functionalities.
				\end{itemize} 
				\item \textsc{MI} is used to handle the geographical issues. The system needs to:
				\begin{itemize}
					\item \textbf{Retrieve} the name of the street where the violation occurred by using the provided position
					\item \textbf{Highlight} the safety of the streets and of municipality's areas
				\end{itemize}
			\end{itemize}
			\item \textbf{Right hand side:} interfaces that enable the system to send and receive data to the authorities. 
			
			In particular when:
			\begin{itemize}
				\item The reported violation has been accepted by the system and enriched with the metadata that can be useful for the authority. In this case the correct information is sent to the authority interested in the area that contains the related position.
				\item The municipality provides information about the accidents that occur in its area. Received data is crossed by the system to determine the safety of its streets. Thanks to this additional operation the application is also able to determine the best interventions for unsafe areas and suggest them to its users.
			\end{itemize}
		\end{itemize}
	
	\subsubsection{Model Structure}
	The static analysis now continues to define the internal structure of the system, in particular with a high-level class diagram that shows the most important objects and their relations in order to achieve the \hyperref[sec:goals]{\textcolor{blue}{goals}}.\\
	
	The main objects in the UML diagram (\autoref{fig:classDiagram}) are:
	\begin{itemize}
		\item \textbf{Customer:} the system has to track two types of users. The distinction, in fact, is fundamental to recognize the municipality providing accident's data but also to give a different level of visibility to the information asked by a request.
		\item \textbf{User:} identifies a citizen with all the data he will provide in its registration. Users are the clients of the application that report parking violations and look for \textbf{general} statistics related both to the frequency of parking violations in a street or to the safety of an area and its related suggested intervention.
		\item \textbf{Authority:} identifies the authority/municipality with all the data related to its recognition. Authorities are the clients of the application that receive the reported violations by the users but they also: ask for \textbf{specific} statistics and provide data about accidents that will be crossed by the system.
		\item \textbf{Registration:} is used to authenticate the customer in order to perform actions he can do only if he is recognized as a user or an authority.
		\item \textbf{Violation:} represents a general traffic violation. This class is thought to be used in a future extension of the system in case more violations are considered.
		\item \textbf{ParkingViolation:} is the result of a report provided by a user. In this way the application considers all the possible information that is filled whenever an infraction is going to be reported. As we see in the diagram the class contains: multiple images for a better help to both the image recognition algorithm and the authorities; the position retrieved by the GPS that will be used to find the name of the street; the plate's number and all other metadata.
		\item \textbf{Accident:} will be used to store the data received by the municipality. No more specific description can be given here because only when interfacing with the municipality we will be able to know how the data has to be managed.
		\item \textbf{Position:} stores the coordinates of where the parking violation occurred. Positions of each accident will be used to retrieve the street thanks to the functionalities provided by the MI and thus obtain statistics of each street and area.
		\item \textbf{Street:} is one of the most important object to be managed. Each recognized street will be added to the system to perform its basic and advanced functionalities. A street considers also how many infractions happened in it; this is thought to simplify the very computing mining and crossing algorithms.
		\item \textbf{Area:} is another important object, in particular for the advanced functionality. An area contains the streets relative to its limits and is managed by the municipality that will receive all the violations reported in it. Areas are also critical for the safety they will encounter, to underline this we can see the only method of the diagram (getSafety) that will be used later to define the related \textbf{state diagram}.
		\item \textbf{Vehicle:} vehicles are recognized by the image recognition algorithm in order to provide additional information to the filter. 
		\item \textbf{Car:} are expected to be the most reported type of vehicles because the biggest in the ones considered.
		\item \textbf{Motorcycle:} should be also considered as "critical" vehicles, in fact there are more motorcycles parked in the wrong place than in the right one.
		\item \textbf{Bicycle:} our system is supposed to deal also with bicycles. Obviously it will be much more difficult for the authorities to retrieve their owners but they can be used by the application to suggest possible interventions in the areas where they happen to be cumbersome.
		\item \textbf{Request:} is the general class representing the interaction of a user with the system whenever statistics about the frequency, safety or suggestion is asked by the user. In order to answer with the correct data it will be important to retrieve the user who sends the request to provide him the right visibility.
		\item \textbf{BasicRequest:} is the request that deals with the basic functionality. Thanks to it the application will be able to capture the filters selected by the user and answer correctly.
		\item \textbf{AdvancedRequest:} is the request that deals with the advanced functionality. Thanks to it the application will be able to capture the requirements of the user, that in this case can be: the safety of a municipality's area or the suggestion for a possible intervention.
	\end{itemize}
	
	\begin{figure}[h!]
		\centering
		\includegraphics[width=\paperwidth - 4cm, angle=90]{/diagrams/classDiagramModel.png}
		\caption{\label{fig:classDiagram}High-level model structure}
	\end{figure}

\subsection[Product Functions]{\hyperlink{toc}{Product Functions}}

\subsection[User Characteristics]{\hyperlink{toc}{User Characteristics}}

\subsection[Domain Assumptions]{\hyperlink{toc}{Domain Assumptions}}

\subsection[The World and the Machine]{\hyperlink{toc}{The World and the Machine}}