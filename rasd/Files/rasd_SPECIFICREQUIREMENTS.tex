\section[Specific Requirements]{\hyperlink{toc}{Specific Requirements}}

\subsection[External Interface Requirements]{\hyperlink{toc}{External Interface Requirements}}
	\label{sec:externalInterfaceRequirements}
	%todo SafeStreets offers an interface to its customers to provide them its basic and advanced functionalities. All the data needed to authenticate the users will be managed inside the system as well as the information related to the violations in order to be mined and crossed whenever needed.
	\subsubsection[User Interfaces]{\hyperlink{toc}{User Interfaces}}
	\subsubsection[Hardware Interfaces]{\hyperlink{toc}{Hardware Interfaces}}
	\subsubsection[Software Interfaces]{\hyperlink{toc}{Software Interfaces}}
	\subsubsection[Communication Interfaces]{\hyperlink{toc}{Communication Interfaces}}

\subsection[Functional Requirements]{\hyperlink{toc}{Functional Requirements}}

\subsection{Use cases description}
	\subsubsection{User}
	
		\paragraph{Registration}
		\begin{longtable}{p{0.25\linewidth}p{0.75\linewidth}}
			\toprule
			\textbf{Name} & \textbf{Registration} \\
			\midrule
			\textbf{Actors} & User \\
			\midrule
			\textbf{Entry conditions} & The application has started \\
			\midrule
			\textbf{Flow of events} & 
			\begin{enumerate}
				\item The user chooses the sign up option
				\item The user selects the 'user' registration type from a list
				\item The user chooses a username and a password
				%todo le authority scelgono regione e paese e il sistema genera username come nomepaese+codiceF205
				\item The user submits the form
				\item The system checks the username to be unique
				\item The system saves the user data
			\end{enumerate} \\
			\midrule
			\textbf{Exit conditions} & The user is registered in the system\\
			\midrule
			\textbf{Exceptions} & 
			\begin{itemize}
				\item If the username inserted by the user is already used by another user, or if the username is reserved to the authorities, the system displays an error message asking the user to insert a different one
			\end{itemize} \\
			\bottomrule
			\caption{\emph{Registration} use case description}
		\end{longtable}
	
		\paragraph{Login}
		\begin{longtable}{p{0.25\linewidth}p{0.75\linewidth}}
			\toprule
			\textbf{Name} & \textbf{Login} \\
			\midrule
			\textbf{Actors} & User \\
			\midrule
			\textbf{Entry conditions} & The application has started \\
			\midrule
			\textbf{Flow of events} & 
			\begin{enumerate}
				\item The user chooses the login option
				\item The user chooses the 'user' login type from a list
				\item The user inserts his username
				\item The user inserts his password
				\item The user submits the form
				\item The system checks the username to be existing
				\item The system checks the password to be right for that username
				\item The system notifies the user that login is successful
			\end{enumerate} \\
			\midrule
			\textbf{Exit conditions} & The user is logged in\\
			\midrule
			\textbf{Exceptions} & 
			\begin{itemize}
				\item If the username is not recognized by the system, that means that the user is not registered yet, or the username is incorrect. The system notifies the user and asks him to insert his username again
				\item If the inserted password is wrong, the system notifies the user and asks him to insert his password again			
			\end{itemize} \\
			\bottomrule
			\caption{\emph{Registration} use case description}
		\end{longtable}
	
		\paragraph{Report violation}
		\begin{longtable}{p{0.25\linewidth}p{0.75\linewidth}}
			\toprule
			\textbf{Name} & \textbf{Report violation} \\
			\midrule
			\textbf{Actors} & User \\
			\midrule
			\textbf{Entry conditions} & The user is logged in \\
			\midrule
			\textbf{Flow of events} & 
			\begin{enumerate}
				\item The user selects the report violation option
				\item The system retrieves the GPS location
				\item The user chooses the option to take pictures
				\item The user takes some pictures through the application
				\item The user selects the pictures he wants to send
				\item The user optionally inserts the license plate number
				\item The user chooses the type of violation from a list
				\item The user chooses the type of vehicle from a list
				\item The user chooses the option to confirm
				\item The system receives the sent data
				\item The system runs an algorithm to read the license plate, with the help of the information provided by the user
				\item The system retrieves the name of the street from the GPS location
				\item The system stores the violation report
			\end{enumerate} \\
			\midrule
			\textbf{Exit conditions} & The information about the violation is stored\\
			\midrule
			\textbf{Exceptions} & 
			\begin{itemize}
				\item If the system fails to retrieve the GPS location, the user is notified and the application shows the home page
				\item If the system fails to read the license plate, or what it reads does not match the information provided by the user, the field will be left empty
			\end{itemize} \\
			\bottomrule
			\caption{\emph{Report violation} use case description}
		\end{longtable}

		\paragraph{Find streets with the highest number of violations}
		%todo{problema della diversa visibilità: differenziamo i casi? facciamo che serve il login per tutto?}
		%todo{le evienziamo su mappa? o restituiamo una lista?}
		%todo{assumption che il programma funziona solo in Italia?}
		\begin{longtable}{p{0.25\linewidth}p{0.75\linewidth}}
			\toprule
			\textbf{Name} & \textbf{Find streets with the highest number of violations} \\
			\midrule
			\textbf{Actors} & User \\
			\midrule
			\textbf{Entry conditions} & The application has started \\
			\midrule
			\textbf{Flow of events} & 
			\begin{enumerate}
				\item The user selects the 'streets with highest number of violation' option
				\item The user chooses the region he is looking for from a list
				\item The user chooses the city from a list
				\item The user chooses the types of violation that he wants to include
				%maybe a checkbox
				\item The user chooses the time slot
				\item The user confirms the query and sends it
				\item The system returns a list of the streets ordered by the highest number of violations, with the actual number next to the name of the street, corresponding to the chosen city and the chosen types of violation, in the specified time slot
			\end{enumerate} \\
			\midrule
			\textbf{Exit conditions} & The list of the streets is shown to the user \\
			\midrule
			\textbf{Exceptions} & 
			\begin{itemize}
				\item 	If no type of violation is selected, the system notifies the user and wait for him to insert at least one	
			\end{itemize} \\
			\bottomrule
			\caption{\emph{Find streets with the highest number of violations} use case description}
		\end{longtable}
		
		\paragraph{Find most dangerous vehicles}
		\begin{longtable}{p{0.25\linewidth}p{0.75\linewidth}}
			\toprule
			\textbf{Name} & \textbf{Find dangerous vehicles} \\
			\midrule
			\textbf{Actors} & User \\
			\midrule
			\textbf{Entry conditions} & The application has started \\
			\midrule
			\textbf{Flow of events} & 
			\begin{enumerate}
				\item The user selects the 'most dangerous vehicles' option
				\item The user selects the region and the city from a list, or selects everywhere
				\item The user selects the types of violation he wants to include
				\item The user confirms the query and sends it
				\item The system returns a list of the types of vehicles, ordered by the highest number of violations they committed
			\end{enumerate} \\
			\midrule
			\textbf{Exit conditions} & The list is shown to the user\\
			\midrule
			\textbf{Exceptions} & 
			\begin{itemize}
				\item 	If no type of violation is selected, the system notifies the user and wait for him to insert at least one	
			\end{itemize} \\
			\bottomrule
			\caption{\emph{Find most dangerous vehicles} use case description}
		\end{longtable}
		
		\paragraph{Find urgent interventions}
		\begin{longtable}{p{0.25\linewidth}p{0.75\linewidth}}
			\toprule
			\textbf{Name} & \textbf{Find urgent interventions} \\
			\midrule
			\textbf{Actors} & User \\
			\midrule
			\textbf{Entry conditions} & The application has started \\
			\midrule
			\textbf{Flow of events} & 
			\begin{enumerate}
				\item The user selects the 'urgent interventions' option
				\item The user selects the region and the city from a list
				%todo Magari facciamo anche filtrare per tipi di interventions
				\item The user confirms the query and sends it
				\item The system returns a list of the most urgent interventions in the selected city, each with their respective street
				%todo comunque in qualche modo i tipi di intervento vanno modellati
			\end{enumerate} \\
			\midrule
			\textbf{Exit conditions} & The list is shown to the user\\
			\midrule
			\textbf{Exceptions} &  \\
			\bottomrule
			\caption{\emph{Find urgent interventions} use case description}
		\end{longtable}
	
		\paragraph{Find unsafe streets}
		%todo ci vuole anche il caso in cui scelgo un percorso da fare e ss mi dice le strade pericolose in quel percorso
		\begin{longtable}{p{0.25\linewidth}p{0.75\linewidth}}
			\toprule
			\textbf{Name} & \textbf{Find unsafe streets} \\
			\midrule
			\textbf{Actors} & User \\
			\midrule
			\textbf{Entry conditions} & The application has started \\
			\midrule
			\textbf{Flow of events} & 
			\begin{enumerate}
				%todo ma quanto bisogna specificare quello che fa il sistema in tutto questo?
				\item The user selects the 'unsafe streets' option
				\item The user selects the either the 'city' option or the 'route' option
				\item The user chooses the region and the city from a list if 'city' option is chosen, otherwise he enters start and end points of the route
				\item The user selects the types of violation he wants to include
				\item The user confirms the query and sends it
				\item The system returns a list with all the unsafe streets included in the query
			\end{enumerate} \\
			\midrule
			\textbf{Exit conditions} & The list is shown to the user\\
			\midrule
			\textbf{Exceptions} & 
			\begin{itemize}
				\item If no city is selected when 'city' option is chosen, the system notifies the user and waits for him to insert it
				\item If no start or end points are chosen when 'route' option is selected, the system notifies the user and waits for him to insert them
				\item If
			\end{itemize} \\
			\bottomrule
			\caption{\emph{Find unsafe streets} use case description}
		\end{longtable}
		
	\subsubsection{Authority}
	
	
\subsection[Performance Requirements]{\hyperlink{toc}{Performance Requirements}}

\subsection[Design Constraints]{\hyperlink{toc}{Design Constraints}}
	\subsubsection[Standards Compliance]{\hyperlink{toc}{Standards Compliance}}
	\subsubsection[Hardware Limitations]{\hyperlink{toc}{Hardware Limitations}}
	\subsubsection[Any other Constraint]{\hyperlink{toc}{Any other Constraint}}

\subsection[Software System Attributes]{\hyperlink{toc}{Software System Attributes}}
	\subsubsection[Reliability]{\hyperlink{toc}{Reliability}}
	\subsubsection[Availability]{\hyperlink{toc}{Availability}}
	\subsubsection[Security]{\hyperlink{toc}{Security}}
	\subsubsection[Maintainability]{\hyperlink{toc}{Maintainability}}
	\subsubsection[Portability]{\hyperlink{toc}{Portability}}
