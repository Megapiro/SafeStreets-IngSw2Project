\section[Introduction]{\hyperlink{toc}{Introduction}}
\label{sec:introduction}
\subsection[Purpose]{\hyperlink{toc}{Purpose}}
	\label{sec:purpose}
	This document completely describes the system in terms of functional and non-functional requirements and is used as a contractual basis between the customer and the developer. The structure follows the template studied during lessons and is integrated with specific documents relative in particular to: the template used \cite{IEEErasd}, the requirements engineering process \cite{IEEEre} and the categorization of the phenomena \cite{W&M}. Hence this document is the result of the requirements elicitation and the analysis activities paired with a specific description aimed to precise the behavior of the system.
	
\subsection[Context]{\hyperlink{toc}{Context}}
	SafeStreets is a \emph{crowd-sourced} application that intends to provide a way to notify authorities when a traffic violation occurs. Nowadays almost everyone uses a device for most of the activities and this gives the opportunity to build products whose knowledge is based on the concept of \emph{crowdsourcing}. Thanks to the system we are going to describe, in fact, each person will be able to report a violation to the authority in a very easy and fast way. The big amount of data SafeStreets expects to receive also allows it to provide to its customers additional functionalities relative to statistics and important considerations about safety. However the main purpose of the application still remains to come up with a simple and effective way to report traffic violations, a problem that is getting more and more important in particular in big cities where authorities are unable to deal with all the infractions that may occur. Without loss of generality we will consider the system to work only in Italy. This assumption is made in particular to obtain a way to recognize authorities as such using a technology already widely used (PEC, Posta Elettronica Certificata). The correct recognition of the authorities allows also to manage correctly one of the most difficult problem that SafeStreets has to deal with: the \emph{law}. In this way we accomplish both the problems of the security of the data relative to the violations and the one related to the correctness of the violations reported by the users.
	
\subsection[Scope]{\hyperlink{toc}{Scope}}
	The aim of the system is to achieve with \emph{crowdsourcing}, a simple way to notify traffic violations and use this information to retrieve interesting data for both its customers: users and authorities. Hence we can think as the main objective of SafeStreets the one of the notification. Then, the other functionalities in the basic and advanced services will be described.\\
	
	The system will describe the notification functionality in terms of \emph{parking violations} but is thought to include further extensions in order to deal with the other types of violations. Users are the customers allowed to report a violation and authorities will be notified whenever a new one will be reported. This process takes place entirely inside the application where both the users and authorities, once recognized, will only be able to benefit of the services related to their role. As we have already said, SafeStreets is a crowd-sourced application; thanks to this property the big amount of data that is going to be managed will be used to provide additional functionalities based on the processes of mining and crossing this information. We call the \emph{basic functionality} the one related to the data mining used to retrieve statistical data interesting for both the users and authorities. It is important to highlight that this functionality considers the role of each customer in order to provide him a way for retrieving the data with a specific level of visibility. As completely described in the section related to the functionalities provided by the application (\blueRef{sec:productFunctions}), the authorities in fact benefit of an additional service that allows them to retrieve the information about the reported parking violations in a more detailed way (for example with the entire description of an infraction that contains also the data about who reported it and the plate of the vehicle). The \emph{advanced functionality} instead is related to the concept of \textbf{safety} that can be attributed to a street. This functionality is based on the crossing process between the data stored in SafeStreets system that are related to the violations and the one of the accidents possibly provided by a municipality. The safety of each street our application is able to retrieve is strictly bound to an additional service that allow customers to find a suggestion for a possible intervention for a street that has been marked as \emph{unsafe}.\\
	
	 Now that we have a first description of what SafeStreets aims to achieve, we can have a look at the \textbf{goals} that have been chosen in order to accomplish these functionalities. Further in this document, we will have the precise definition of each goal with the prove of their satisfaction described in terms of requirements and assumptions (section \blueRef{sec:functionalRequirements}). The description will be carried on first with the identification of the phenomena (section \blueRef{sec:worldMachine}), and then with the additional strict relationship with the part that describes the requirements used for the goals and the use case diagrams (section \blueRef{sec:useCases} and traceability matrix \blueRef{tab:traceabilityMatrix}). 
	
	\subsubsection[Goals]{\hyperlink{toc}{Goals}}
		\label{sec:goals}
		\begin{enumerate}[label=\textbf{G\arabic*}]
			\item \label{goal:notification} Users should be able to notify authorities when traffic violations occur, in particular parking violations.
			\item \label{goal:mining} Users and authorities should be able to mine the information stored by SafeStreets, with different levels of visibility.
				\begin{enumerate}[label=\textbf{G2\Alph*}]
					\item \label{goal:miningA} Users and authorities should be able to know where the highest number of violations occur.
					\item \label{goal:miningB} Users and authorities should be able to know what types of vehicle make the most violations.
					\item \label{goal:miningC} Authorities should be able to consult every violation report sent by users.
				\end{enumerate}
			\item \label{goal:safety} Users and authorities should be able to know which streets are safe and which ones are not.
			\item \label{goal:intervention} Users and authorities should be able to know the possible interventions that could be done in a city.
		\end{enumerate}
	
\subsection[Glossary]{\hyperlink{toc}{Glossary}}
	\label{sec:glossary}
	
	\subsubsection[Definitions]{\hyperlink{toc}{Definitions}}
		\begin{itemize}
			\item \textbf{DBMS:} a software system that allows to manage efficiently the data stored in the databases of the system.
			\item \textbf{Posta Elettronica Certificata:} a technology that allows to send email with a legal approach.
			\item \textbf{Crowdsourcing:} a sourcing model in which individuals or organizations obtain goods and services from a large, relatively open and often rapidly-evolving group of internet users.
		\end{itemize}
	
	\subsubsection[Acronyms]{\hyperlink{toc}{Acronyms}}
		\begin{itemize}
			\item \textbf{DBMS:} Database Management System
			\item \textbf{EMS:} Email Management System
			\item \textbf{PEC:} Posta Elettronica Certificata
			\item \textbf{API:} Application Programming Interface
			\item \textbf{IRI:} Image Recognition Interface
			\item \textbf{MI:} Map Interface
			\item \textbf{ACI:} Authority Common Interface
			\item \textbf{GPS:} Global Positioning System
			\item \textbf{OS:} Operating System
		\end{itemize}
	
	\subsubsection[Abbreviations]{\hyperlink{toc}{Abbreviations}}
		\begin{itemize}
	        \item \textbf{i.i.f.:} if and only if
	        \item \textbf{a.k.a.:} also known as
	        \item \textbf{e.g.:} exempli gratia
			\item \textbf{G:} goal
			\item \textbf{DA:} domain assumption
			\item \textbf{R:} requirement
			\item \textbf{US:} user scenario
			\item \textbf{AS:} authority scenario
		\end{itemize}
		
\subsection[Document Structure]{\hyperlink{toc}{Document Structure}}
	The document is structured in a double linked way in order to provide an easier and quicker navigation in particular for the parts where several abbreviations are used and the entire text would not fit in the layout (the reader can try this by looking at the requirements in the traceability matrix \blueRef{tab:traceabilityMatrix}). The aim is to give a description that interconnects the most interesting parts of the document that are also related in a theoretical point of view: \textbf{\hyperref[sec:worldMachine]{World and Machine}},
	\textbf{\hyperref[sec:goalSatisfaction]{Goals and Requirements}}  and \textbf{\hyperref[sec:useCases]{Use Cases}.}\\
	
	Moreover the document is structured as now briefly described:
	\begin{enumerate}
		\item \textbf{\hyperref[sec:introduction]{Introduction}:} gives a first description of the problem and describes the purpose of the SafeStreets system. Goals are also highlighted to enforce the previous shallow description; the section ends with the glossary.
		
		\item \textbf{\hyperref[sec:overallDescription]{Overall Description}:} starts with the product perspective where first the system is highlighted from the outside and then from the inside with a high-level description of its structure. State diagrams are then used to clarify the behavior of the most critical objects identified in the modeling process and then product functions are ready to be precisely described. The section ends with the identification of the important phenomena for the problem that are now clearly described with the World and Machine paradigm.
		
		\item \textbf{\hyperref[sec:specificRequirements]{Specific Requirements}:} in this section, requirements are precisely described starting with the ones of the interfaces that SafeStreets uses to provide its services to the external world. Functional requirements, where the satisfaction of the goals is proved thanks to the requirements, and the domain assumptions previously defined, are the core of this section. Use cases are also important in particular to highlight their strict relation with the requirements also highlighted with the traceability matrix.
		
		\item \textbf{\hyperref[sec:formalAnalysisUsingAlloy]{Formal Analysis Using Alloy}:} includes the model that is described formally thanks to the alloy language \cite{Alloy}. This section highlights the most critical aspects of the entire problem and proves their satisfaction in specific worlds generated for this purpose.
		
		\item \textbf{\hyperref[sec:effortSpent]{Effort Spent}:} this section has been used to keep track of the hours spent to complete this document. The first table defines the hours spent together while taking the most important decisions, the seconds instead contain the individual hours.
		
		\item \textbf{\hyperref[sec:references]{References}:} includes all the references used to define the document.
	\end{enumerate}
